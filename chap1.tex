\chapter{The Basic Concepts}
\section{Thermodynamic Systems}

\begin{enumerate}
\item[$\bullet$] \tb{Isolated systems} do not exchange energy or matter with the exterior.
\item[$\bullet$] \tb{Closed systems} exchange energy with the exterior but not matter.
\item[$\bullet$] \tb{Open systems} exchange both energy and matter with the exeterior.
\end{enumerate}

The \tb{state} of a system is specified in terms of macroscopic \tb{state variables} such as volume V, pressure p, termperature T,
mole number of the chemical constiuents $N_k$. The two laws of thermodynamics are founded on the concepts of energy U, and entropy
S, which are \tb{functions of state variables}. It is convenient to classify thermodynamic variables into two categories: variables
such as volume and mole number, which are proportional to the size of the system, are called \tb{extensive variables}. Variables
such as temperature T and pressure p, that specify a local property, which are independt of the size of the system, are called \tb{
intensive variables}. If the temperature is not uniform, heat will flow until the entire system reaches a state of uniform 
temperature, the state of \tb{thermal equilibrium}.

In the state of thermal equilibrium, the values of total interal energy U and entropy S are specified by 
\begin{equation}
U=U(T,V,N_k), S=S(T,V,N_k)
\end{equation}
The value of extensive variable can also be specified by other extensive variables:
\begin{equation}
U=U(S,V,N_k), S=S(U,V,N_k)
\end{equation}
Intensive variables can be expressed as derivatives of one extensive variable with respect to anouther. For example, $T=\left(
\partial U/\partial S\right)_{V,N_k}$

\section{Equilibrium and Nonequilibrium Systems}
If a physical system is isolated, its state evolves \ti{irreversibly} towards a time-invariant state in which we see no further
physcial or chemical change in the system. It is a state characterized by a uniform temperature throughout the system. This is the
state of \tb{thermodynamic equilibrium}.

The evolution of a state of a state towards the sate of equilibrium is due to irreversible process. At equilibrium, these processes 
vanish. Thus, a nonequilibrum state can be characterized as a state in which irreversible processes drive the system to the state
of equilibrium.  

If a system A is in equilibrium with system B and if B is in equilibrium with system C, then it follows that A is in equilibrium 
with C. This "transitivity" of the state of equilibrium is some times called the \tb{zeroth law}. Thus, equilibrium systems have
one uniform temperature and for these systems there exist state functions of energy and entropy. Uniformity of temperature, however,
is not a requirement for the entropy or energy of a system to be well defined. For \tb{nonequilibrium systems} in which the 
temperature is not uniform but is well defined locally, we can define density of thermodynamic quantities such as energy and entropy.
Thun the system's energy and entropy can be expressed as $S=\int_Vs(T(x),n_k(x))dV, U=\int_Vu(T(x),n_k(x))dV,N=\int_Vn_k(x)dV$. So
in nonuniform systems, the total energy U is no longer a functions of other extensive variables such as S, V and N.

\section{Temperature, Heat and Quantitative Laws of Gases}
\begin{definition}
\tb{1Pa}=1N$\mathrm{m}^{-2}$, \tb{1bar}=$10^5$\tb{Pa}, (T/K)=(T/ \textcelsius ) + 273.15
\end{definition}
\tb{Heat} was initially thought to be an indestructible substance called the \tb{caloric}. However, in the 19th century it was establised
that heat was not an indestructable caloric but a form of energy that can convert to other froms. However, we still use \tb{caloric} to 
measure heat.

A series of experiments setup up the \tb{law of ideal gases} $pV=NRT, R=8.31441JK^{-1}mol^{-1}$. For a mixture of ideal gases we have the
\tb{Dalton law of partial pressures}, if $p_k$ is the partial pressure due to component k, then $p_kV=N_kRT$. Since a gas expanding into
vacuum does not do any work during the processes of expansion, its energy does not change. The fact that temperature does not change 
during this expansion into vacuum while the volume and pressure do change, implies that the energy of a given amount of ideal gas depends
only on its temperature, not on volume or presure. Thus the energy of the ideal gas can be written as $U(T,N)=NU_m(T)$, which $U_m$ is the
total internal energy per mole which can be expressed as $U_m=cRT+U_0$, where $U_0$ is constant, for monatomic gases such as He and Ar,
c=3/2, for diatomic gases such as $N_2,O_2$, c=5/2. In the ideal gas equation, the volume of ideal gas can be compressed to infinite small.
However, it is contradiction to physical intuition. In 1873, van der Waals proposed an equation in which he incorporated the effects of
attractive forces between molecules and molecular size on the pressure and volume of gas. 
\begin{equation}
(p+aN^2/V^2)(V-Nb)=NRT
\end{equation}

\section{States of Matter and the van der Walls Equation}
In thermodynamics the various states of matter - solid, liquid, gas - are often refered to as \tb{phases}. Joseph Black and James Watt 
discovered another interesting phenomenon associated with the changes of phase: at the melting or the boiling temperature, the heat supplied
to a system does not produce an increase in temperature; it only converts the substance from one phase to another. The van der Waals equation 
also exhibits a \tb{critical temperature} $T_c$: if the temperature T is greater than $T_c$ the p-V curve is always single-valued,
indicateing there is no transition to the liquid state. The van der Waals equation has two extrema for $T<T_c$. Below $T_c$, as T increases,
these two extrema move closer and finally coalesce at $T=T_c$. Above the ciritical temperature there is no phase transition from a gas to a liquid.
The pressure and volume at which this happens are called the \tb{critical pressure} $p_c$ and \tb{critical volume} $V_c$. We note that if
we regard $p(v,T)$ as function of V, then at the cirtical point $T=T_c,p=p_c,V=V_c$. We have an inflection point, for the first and second
derivatives vanish. 
\begin{equation}
\left(\frac{\partial p}{\partial V}\right)_T=0 \qquad \left(\frac{\partial^2 p}{\partial V^2}\right)_T=0
%\(\frac{\partial p}{\partial V}\)_T=0 \qquad \(\frac{\partial^2 p}{\partial V^2}\)_T=0
\end{equation}
The solution is expressed as follows
\begin{equation}
T_c=\frac{8a}{27Rb}\qquad p_c=\frac{a}{27b^2}\qquad V_mc=3b
\end{equation}